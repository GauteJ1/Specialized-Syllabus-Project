In this subsection, we jointly analyze measurements from multiple sources in order to reveal underlying patterns across them.
The datasets consist of fluorescence spectroscopy EEMs $\mathcal{X} \in \mathbb{R}^{28 \times 251 \times 21}$, identical to part one, as well as nuclear magnetic resonance (NMR) spectroscopy $\mathcal{Y}\in \mathbb{R}^{28 \times 13324 \times 8}$ and liquid chromatography-mass spectrometry (LC-MS) measurements $Z \in \mathbb{R}^{28 \times 168}$, all from the same 28 mixtures.

We will jointly analyze the three datasets using the AO-ADMM Data Fusion Framework 
\cite{framework}, demonstrate the uniqueness property of the model by comparing different runs, and finally plot the discovered factors.


\subsection*{Background and Method}
As in part 1, our first task is to handle the missing values in $\mathcal{X}$.
We note that neither $\mathcal{Y}$ nor $Z$ contain any missing values.
While the \texttt{cp\_wopt} function we used in part 1 has a simple way of handling missing values internally by weighting the loss function, this is not the case for the method used in part 2.
Hence, we need to handle these missing values before fitting the models.
There are various methods to do this, but we chose to use the best model from part 1 to replace the missing values.
This seems like a good solution, as the model gives replacement values based on the entire rest of the dataset, instead of a weighted average of some selected values, as is the case for some alternatives.
Instead of using $\ten{X}$ directly, we fitted the coupled model using $\ten{X}^*$ defined as in \eqref{eq:Xx} instead, where $*$ denotes the element-wise Hadamard product, $\ten{\hat{X}}$ is the best $R=3$ model from part 1, $\ten{W}$ is defined as in \eqref{eq:weight_matrix}, and $\ten{W}'$ is defined as $\ten{W}'_{i,j,k} = 1 - \ten{W}_{i,j,k}$.
\begin{equation}
    \ten{X}^* = \ten{W}*\ten{X} + \ten{W}' *\ten{\hat{X}}
    \label{eq:Xx}
\end{equation}

We analyze the data jointly using the AO-ADMM Data Fusion Framework\cite{framework}.
This framework utilizes Alternating Optimization (AO) and the Alternating Direction Method of Multipliers (ADMM).
For more information about these algorithms, see \textcite{framework}.
We use the Frobenius tensor norm as the loss function on all three matrices, and L-BFGS-B as the optimization algorithm.



Based on the information given about the underlying components, we use the coupling set-up described in case 3b in \textcite{framework}.
As the total number of components is 6 (3 across all tensors, 1 across $\ten{Y}$ and $Z$, 1 only in $\ten{Y}$ and one only in $Z$), we let the matrix $\Delta_1 \in \R^{28 \times 6}$.
The factor 28 comes from all 28 mixtures being coupled across all three tensors/matrices.
By letting $C_{1,1}$ denote the factor matrix of the first mode in $\ten{X}$, $C_{2,1}$ the first mode in $\ten{Y}$, and $C_{3,1}$ in $Z$, we have $C_{i, 1} = \Delta_1 H_{i,1}$ for each $i=1,2,3$, where $H_{i,1}$ are defined for each $i$ as in \eqref{eq:hmat}.

\begin{equation}
    H_{1,1} = \begin{bmatrix}
        1 & 0 & 0 \\
        0 & 1 & 0 \\
        0 & 0 & 1 \\
        0 & 0 & 0 \\
        0 & 0 & 0 \\
        0 & 0 & 0
    \end{bmatrix}, \quad
    H_{2,1} = \begin{bmatrix}
        1 & 0 & 0 & 0 & 0 \\
        0 & 1 & 0 & 0 & 0 \\
        0 & 0 & 1 & 0 & 0 \\
        0 & 0 & 0 & 1 & 0 \\
        0 & 0 & 0 & 0 & 1 \\
        0 & 0 & 0 & 0 & 0
    \end{bmatrix}, \quad
    H_{3,1} = \begin{bmatrix}
        1 & 0 & 0 & 0 & 0 \\
        0 & 1 & 0 & 0 & 0 \\
        0 & 0 & 1 & 0 & 0 \\
        0 & 0 & 0 & 1 & 0 \\
        0 & 0 & 0 & 0 & 0 \\
        0 & 0 & 0 & 0 & 1
    \end{bmatrix}
    \label{eq:hmat}
\end{equation}

Note that from the matrices in \eqref{eq:hmat}, we see that the $\ten{X}$ will be modeled by the first 3 factors, $\ten{Y}$ by components 1 through 5, and $Z$ by components 1 to 4 and 6, as described in the task.

We also note that the matrix $C_{1,1} = \Delta_1 H_{1,1}$ will be of size $28 \times 3$, as expected.
Similarly, both $C_{2,1}$ and $C_{3,1}$ will be of size $28 \times 5$.

As in part 1, we will perform multiple fitting runs with randomly drawn initial conditions in order to obtain the best model possible.
We also compare the models that have (almost) equal loss to the loss of the best model, in order to demonstrate uniqueness of the factors.

